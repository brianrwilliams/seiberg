\documentclass[11pt]{amsart}

\usepackage{macros}
\usepackage{lscape,pdflscape}

\setlength\parindent{0pt}

%\newcommand{\sslash}{\mathbin{/\mkern-6mu/}}

\renewcommand{\op}{\operatorname}
\newcommand{\bm}{\mathbf{m}}
\newcommand{\clie}{C^\bu_{\rm Lie}}
\newcommand{\cbry}{C^\bu_{\rm Bry}}
%\DeclareMathOperator{\Sym}{Sym}
\newcommand{\csym}{\widehat{\Sym}}
\DeclareMathOperator{\SL}{SL}
\DeclareMathOperator{\Mat}{Mat}
\DeclareMathOperator{\Proj}{Proj}

%\addbibresource{refs}

\usepackage{subfiles}


\begin{document} 

\title{Notes by Owen}

\maketitle

\section{2 November 2024}

\subsection{About the derived geometry of holomorphic gauged linear sigma models (hglsm)}

Let me talk about the finite-dimensional setting before talking about field theories.

Let $V$ be an affine derived space, e.g., a cochain complex,
and suppose a reductive algebraic group $G$ acts on it.
There are three spaces whose analogs play a role in field theory:
\begin{itemize}
\item the stacky quotient $V/G$,
\item the formal neighborhood of the origin in the stacky quotient, which I'll denote as $\widehat{V}/\fg$, and
\item the affinization of $V/G$, which I'll denote by $V \sslash G$.
\end{itemize}
Observe that we have natural maps
\[
\widehat{V}/\fg \to V/G \to V\sslash G
\]
by definition.\footnote{We will denote a stacky and derived quotient with the benign notation of a single slash, as it is the ``correct'' notion. We will use $\sslash$ for GIT quotients (always affine for us).}

The latter two spaces are affine in nature, with dg algebras of functions
\[
\cO(\widehat{V}/\fg) = \clie(\fg, \csym(V^*))
\]
since $\cO(\widehat{V}) = \csym(V^*)$ and
\[
\cO(V\sslash G) = \cO(V)^G = \Sym(V^*)^G.
\]
The composite map $\widehat{V}/\fg \to V\sslash G$ corresponds to the algebra map
\[
\cO(V)^G \to \clie(\fg, \cO(\widehat{V})),
\]
which is not a quasi-isomorphism but is injective cohomologically when $V$ is an ordinary vector space.

We are interested in the holomorphic gauged linear sigma model (hglsm), or gauged $\beta\gamma$ system, whose space of solutions encodes
\[
\Map_{\dbar}(\Sigma, T^*(V/G))
\]
with $\Sigma$ a Riemann surface.

Let us focus on $\Sigma$ a Stein space (i.e., the analytic analog of an affine algebraic curve) 
for simplicity, and because it is the only case relevant when we want to extract a vertex algebra.

We introduce some convenient notation:
\begin{itemize}
\item $\fg^\Sigma$ denotes the dg Lie algebra $\Omega^{0,\bu}(\Sigma) \otimes \fg = \Map_{\dbar}(\Sigma,\fg)$,
\item $G^\Sigma$ denotes the derived group stack $\Map_{\dbar}(\Sigma,G)$, 
\item $(T^*V)^\Sigma$ denotes the derived stack $\Map_{\dbar}(\Sigma,T^*V)$,
and
\item the formal neighborhood of the origin in $T^*(V/G)$ is encoded by the graded Lie algebra
\[
L_0 = \fg \oplus V[-1] \oplus V^*[-2] \oplus \fg^*[-3],
\] 
whose degree 0 component is the Lie algebra $\fg$ and whose remaining components form an abelian subalgebra equipped with the natural action of~$\fg$.
\end{itemize}
We now turn to the derived spaces relevant to the hglsm.

In this setting there are {\em five} spaces to consider:
\begin{itemize}
\item the nonperturbative classical theory encoded by~$\Map_{\dbar}(\Sigma, T^*(V/G))$,
\item the perturbative theory expanded around the map to the origin in $T^*(V/G)$, 
a formal moduli space 
\[
(\widehat{T^* V} \times \fg^*[-2])^\Sigma/\fg^\Sigma
\] 
encoded by the dg Lie algebra $L_0^\Sigma = \Omega^{0,\bu}(\Sigma) \otimes L_0$, 
\item the affinization of~$\Map_{\dbar}(\Sigma, T^*(V/G))$, which I'll denote by 
\[
(T^*V \times \fg^*[-2])^\Sigma \sslash G^\Sigma,
\]
\item and $\Map_{\dbar}(\Sigma,(T^*V \times \fg^*[-2])\sslash G)$.
\end{itemize}
This fourth space is not what we want to work with, 
although it is close to the usual underived thing people often study with glsms.

The fifth space is an interesting hybrid whose derived geometry I don't know how to describe:
it is the affine derived space associated to the dg algebra
\[
\clie(\fg^\Sigma, G; \cO(T^* V \times \fg^*[-2])),
\]
a relative Lie algebra cohomology complex. 
I will denote the associated space
\[
\Spec(\clie(\fg^\Sigma, G; \cO(T^* V \times \fg^*[-2])))
\]
by
\[
((T^* V \times \fg^*[-2])/\fg^\Sigma)\sslash G.
\]
Note that this relative Lie algebra complex sits, in some sense, between the dg algebra
\[
\clie(\Omega^{0,\bu}(\Sigma) \otimes L_0) = \clie(\fg^\Sigma, \cO(T^*\widehat{V} \times \fg^*[-2]))
\]
encoding observables of the perturbative theory
and the dg algebra
\[
\cbry(G^\Sigma, \cO(T^* V \times \fg^*[-2]))
\]
encoding observables of $(T^*V \times \fg^*[-2])^\Sigma \sslash G^\Sigma$,
the affinization of~$\Map_{\dbar}(\Sigma, T^*(V/G))$.
Here $\cbry$ denotes the ``differentiable cohomology for Lie groups'' introduced by Brylinski, 
and which captures the derived invariants for a group like $G^\Sigma$.\footnote{It seems to be unpublished: see \url{https://arxiv.org/abs/math/0011069}. This paper revives and enhances work of Segal and Mitchison.}

The space $\Map_{\dbar}(\Sigma,(T^*V \times \fg^*[-2])\sslash G)$ 

These spaces (aside from the fourth) are related by natural maps
\[
\begin{tikzcd}
(T^* V \times \fg^*[-2])^\Sigma/\fg^\Sigma \arrow[r] & \Map_{\dbar}(\Sigma, T^*(V/G)) \arrow[dd] \arrow[dr]& \\
&&  ((T^* V \times \fg^*[-2])/\fg^\Sigma)\sslash G \arrow[dl, dashed] \\
&(T^*V \times \fg^*[-2])^\Sigma \sslash G^\Sigma&
\end{tikzcd}
\]
where there is a dashed line because I have not confirmed this map exists.

The {\em best} space, from a physics perspective, is the one at the bottom,
as it corresponds to the full algebra of observables,
but I don't know how to compute Brylinski cohomology.
The {\em easiest} space is the perturbative one.
I suspect that we can get our hands on the new, intermediary space without much difficulty.

\subsection{About the {\it classical} factorization and vertex algebras from hglsms}

All the dg algebras described above are manifestly functorial in $\Sigma$, 
so one obtain a prefactorization algebra on, say, $\CC$ by letting $\Sigma$ run over open sets in the complex plane.

These constructions are manifestly holomorphic, 
and they are tame,
so one can obtain a vertex algebra from these constructions.
They can be viewed as BRST-type reductions of the free $\beta\gamma$ system associated with~$V$.

{\bf Guess:} The state space for either $(T^*V \times \fg^*[-2])^\Sigma \sslash G^\Sigma$ or $((T^* V \times \fg^*[-2])/\fg^\Sigma)\sslash G$ is the relative Lie algebra cochains
\[
\clie(\fg[[z]], G; \Sym(V^*[[z]] \oplus V[[z]]))
\]
where $\Sym(V^*[[z]] \oplus V[[z]])$ is the state space of the free $\beta\gamma$ system associated with~$V$.\footnote{If I botched that notation, just substitute the correct thing!}

My guess arises from the fact that the group $G^\Sigma$ becomes $G[[z]]$ when $\Sigma$ becomes the formal disk,
and the ``group'' cohomology for this group ought to coincide with the relative Lie algebra cohomology as there is a short exact sequence
\[
1 \times \fg[[z]]z \to G[[z]] \to G \to 1
\]
with the kernel of the evaluation map at 0 being the unipotent pro-Lie group whose Lie algebra is~$\fg[[z]]z$.

In particular, this guess would suggest that we can get the vertex algebra we want (Brylinski cohomology) using the thing we can compute (relative Lie algebra cohomology).

\subsection{About quantization}

The BV quantization of the charged $\beta\gamma$ system seems like a straightforward variant of stuff we've already done. 

For the perturbative theory (expanding around the map to the origin of $T^*(V/G)$), let $c$ denote the ``ghost field'' in $\Omega^{0,\bu} \otimes \fg[1]$ and let $c^\dagger$ denote the ``antighost field'' in $\Omega^{1,\bu} \otimes \fg^*[-1]$, 
which is a representation for the ghost fields.
Let $\rho: \fg \to \End(V)$ denote the representation on~$V$. 

{\bf Guess:} For quantizing on flat space, the anomaly is a sum 
\[
\Tr_V \int \rho(c) \wedge \partial \rho(c) + \Tr_{\fg^*} \int c \wedge \partial c
\]
up to an overall scale.

(For coupling to background vector fields (i.e., for extending to arbitrary Riemann surfaces), 
there is some anomaly I can't remember, although I know it's a kind of ``vanishing $c_1$'' thing.)

{\bf Question:} Does this BV quantization work for $((T^* V \times \fg^*[-2])/\fg^\Sigma)\sslash G$? That is, on the relative Lie algebra cochains, can we use the BV Laplacian and the quantum corrections to the action?

I would guess {\it yes}.

\subsection{Pay-off}

Our goal, as we discussed, is to formulate ``dualities'' of holomorphic glsms, 
inspired by LG B-models and by compactification of holomorphic twists of 4d dualities among supersymmetric gauge theories.

In particular, it would be useful (eventually) to include
\begin{itemize}
\item superpotentials and
\item further twists (e.g., topological or mixed).
\end{itemize}
Our idea was perhaps to write a rather technical paper on holomorphic glsms, 
and then to expand our survey of the Seiberg duality conjecture with a section about the hglsm situation, where we presumably can quickly read off a proof from our first paper plus Weyl's invariant theory. 

\section{5 November 2024}

\subsection{Recording the facts of invariant theory}

We record here the classic facts.

Let $V = \CC^n$, so $\SL(V) = \SL(n)$. 
Invariant theory aims to describe, as thoroughly as possible, spaces like
\[
(V^{\otimes r} \otimes (V^*)^{\otimes s})^{\SL(V)}
\]
the ``tensor problem'' and
\[
\Sym^k(V^{\oplus r} \otimes (V^*)^{\oplus s})^{\SL(V)}
\]
the ``polynomial problem.''
There are many ways to do this, depending on your mathematical taste.

I will copy a few theorems from Procesi's {\it Lie groups}, but other nice resources include Weyl's {\it The classical groups} or {\it Algebraic Geometry IV}, the Encyclopedia of Mathematical Sciences, by Popov and Vinberg.

\begin{thm}
Let $\Mat(m,n) \cong V^{\oplus m}$ denote the $m \times n$ matrices with right multiplication by $\SL(V)$, and let $\CC[x_{ij}]$ denote
the polynomial ring in entries $(x_{ij})$ of the matrices.
Then
\[
\cO(V^{\oplus m})^{\SL(V)} = \CC[x_{ij}]^{\SL(n)}
\]
is the subring of $\CC[x_{ij}]$ generated by the $n \times n$ minors~$m(i_1,\ldots, i_n)$.
\end{thm}

Here $m(i_1,\ldots, i_n)$ denotes the minor of the $n \times n$ submatrix 
\[
X[i_1,\ldots,i_n]
\]
given by rows $i_1,\ldots,i_n$ of some $X \in \Mat(m,n)$.
These minors satisfy the Pl\"ucker relations, which are quadratic in these generators.
(This assertion about relations is an example of the {\em second} fundamental theorem.)

Geometrically, this theorem says that there is an embedding
\[
\Mat(m,n)\sslash \SL(n) \hookrightarrow \bigwedge^n(\CC^m)
\]
as the decomposable wedge products (i.e., not ``entangled'').
It is precisely the affine cone of the Grassmannian using the standard Pl\"ucker embedding.
That is,
\[
\Spec(\cO(V^{\oplus m})^{\SL(V)}) \cong {\rm Cone}(\Gr(n,m)).
\]
If we take $\Proj$ rather than $\Spec$ of $\cO(V^{\oplus m})^{\SL(V)}$, 
we find
\[
\Proj(\cO(V^{\oplus m})^{\SL(V)}) \cong \Gr(n,m),
\]
so this situation is quite familiar.

The case we really want is the following.

\begin{thm}[FFT for $\SL(n)$]
\label{FFT}
Consider 
\[
\Mat(m,n) \times \Mat(n,p)\cong V^{\oplus m} \oplus (V^*)^{\oplus p},
\]
namely pairs of matrices $(X,Y)$, and let $\CC[x_{ij}, y_{kl}]$ denote functions on such pairs.
Then
\[
\cO(V^{\oplus m} \oplus (V^*)^{\oplus p})^{\SL(V)} = \CC[x_{ij}, y_{kl}]^{\SL(n)}
\]
is the subring of $\CC[x_{ij}, y_{kl}]$ generated by 
\begin{itemize}
\item the $n \times n$ minors $m_X(i_1,\ldots, i_n)$ of~$X$,
\item the $n \times n$ minors $m_Y(j_1,\ldots, j_n)$ of $Y$, and
\item the entries $p_{XY}(i,j)$ of the product~$XY$.
\end{itemize}
\end{thm}

These are examples of {\em first fundamental theorem} of invariant theory for $\SL(n)$.
(These appear in Section 1.2 of Chapter 11 of Procesi.)

We need to show that the relations, satisfied by these generators, are encoded by taking the critical locus of the ``superpotential''
\[
W(X,M,Y) = \Tr(XMY)
\]
where $M \in \Mat(n,n)$.
This claim is equivalent to the {\em second fundamental theorem} of invariant theory for $\SL(n)$,
for which I'm trying to find a convenient reference.

\section{6 November 2024: More about invariant theory}

Donald Trump will be our president again.
What a ride we are choosing to take.

Alright, I found a number of references, although fewer proofs than I expected.
So far the best is Lecture 15 of Kac's course (notes by Laksov at \url{https://people.kth.se/~laksov/notes/invariant.pdf}).
Kac explicates the essential ideas and examples with minimal fuss and archaic notation.

The second fundamental theorem appears as Theorem 14-2.1.

\begin{thm}[SFT for $\SL(n)$]
Consider 
\[
\Mat(m,n) \times \Mat(n,p)\cong V^{\oplus m} \oplus (V^*)^{\oplus p},
\]
namely pairs of matrices $(X,Y)$, and let $\CC[x_{ij}, y_{kl}]$ denote functions on such pairs.
Then the invariant ring
\[
\CC[x_{ij}, y_{kl}]^{\SL(n)}
\]
is the quotient of the free commutative ring 
\[
\CC[m_X(i_1,\ldots, i_n), m_Y(j_1,\ldots, j_n), p_{XY}(i,j)]
\]
on the generators from Theorem~\ref{FFT}
by the ideal generated by
\begin{itemize}
\item the Pl\"ucker relations for the minors~$m_X(i_1,\ldots, i_n)$,
\item the Pl\"ucker relations for the minors~$m_Y(j_1,\ldots, j_n)$, and
\item the relations
\begin{align*}
m_X(i_1,\ldots, i_n)m_Y(i_1,\ldots, i_n) 
&= \det(X[i_1,\ldots,i_n]Y[j_1,\ldots,j_n]) \\
&= \det(p_{XY}(i,j))_{i \in (i_1,\ldots, i_n), j \in (j_1,\ldots, j_n)}
\end{align*}
arising from the fact that the product of determinants equals the determinant of products for square matrices.
\end{itemize}
\end{thm}

\section{10 November: Recording some questions}

It seems to me that there's a puzzle in comparing the derived geometry story with the physics story.
For a hglsm on $d$-manifold with target arising from $V/G$, the solutions to the equations of motion form the moduli space
\[
\Map_{\dbar}(M, T^*[d-1](V/G))
\]
so that translation-invariant solutions on $M = \CC^d$ provide the familiar space
\[
T^*[d-1](V/G) \cong (V \oplus V^*[d-1] \oplus \fg^*[d-2])/G.
\]
I have called this the {\em moduli of classical vacua} at times,
as it matches at least one definition found in the physics literature.

The issue is that this space will have stuff in degree 0 that depends on the Lie algebra $\fg$, 
namely through the ``constant anti-ghosts'' parametrized by $\fg^*[d-2]$.
In particular, for Seiberg duality,
it means the moduli of classical vacua is {\em not} equivalent between the electric and magnetic theories.

It seems to me that there are three options for this:
\begin{enumerate}
\item These antighosts are ``physically irrelevant'' for some reason. (Not obvious to me, although historically they would be ignored by physicists.)
\item Taking the classical truncation, they disappear. (This is true in some special cases but doesn't seem true generally.)
\item Quantization kills them. 
\end{enumerate}
Remarkably, the third option may hold, based on Brian's recollection. 



\end{document}