\documentclass[11pt]{amsart}

\usepackage{macros}
\usepackage{lscape,pdflscape}

\renewcommand{\op}{\operatorname}
\newcommand{\bm}{\mathbf{m}}

%\addbibresource{refs}

\usepackage{subfiles}


\begin{document} 

\title{Notes by Brian}

\maketitle


\section{15 August 2024}

\subsection{Mass terms}

You mentioned mass terms, let me say some things I understand.

Let's start with the holomorphic twist of $\cN=1$ theories.
A ``mass'' term is something which makes sense for chiral multiplets.
It is a special type of superpotential.
The holomorphic twist of the chiral multiplet valued in a vector space $V$ is the $\beta\gamma$ system with action $\int_{\C^2} \beta \dbar \gamma$.
Here $\gamma \in \Omega^{0,\bu}(\C^2) \otimes V$ and $\beta \in \Omega^{2,\bu}(\C^2, V^*)[1]$.
Notice that this theory has a global symmetry by the group $GL(2) \times GL(V)$.
A superpotential breaks this to $SL(2) \times H$ where $H \subset GL(V)$ is some subgroup.
It is given by the data of a polynomial $W \in \cO(V) = S (V^*)$ and a holomorphic volume form on $\C^2$. To preserve translation invariance we take this volume form to be $\d z_1 \d z_2$.
The Lagrangian associated to a superpotential, in the holomorphic twist, is
\beqn
\int \d z_1 \d z_2 \, W(\gamma)  .
\eeqn
Also notice that generically $W$ breaks the cohomological $\Z$ grading to a $\Z/2$ grading.

A `mass term' is a type of superpotential prescribed by the data of symmetric matrix $M \in \Sym^2(V^*) \subset \cO(V)$.
If $\{e_i\}$ is a basis for $V$ with corresponding chiral multiplet fields $\{\gamma_i\}$ then the associated Lagrangian is
\beqn\label{eqn:lagmass}
\int \d z_1 \d z_2 \, M(\gamma) = \int \d z_1 \d z_2 \, M^{ij} \gamma_i \gamma_j .
\eeqn

The simplest example is a single chiral multiplet $\gamma$ with mass term
\beqn\label{eqn:mass1}
\frac12 \int \d z_1 \d z_2 \, m \gamma^2 ,
\eeqn
where $m \in \C$.
A `Dirac mass' makes sense for an even number of chiral multiplets.
For two chiral multiplets $\gamma_1, \gamma_2$ it looks like 
\beqn\label{eqn:mass2}
\int \d z_1 \d z_2 \,  m_{D} \gamma_1 \gamma_2 ,
\eeqn
where again $m_D \in \C$.
Here is an easy fact about these two simple examples.

\begin{prop}
\label{prop:mass}
Suppose $m \ne 0$.
Then the holomorphic twist of the theory of a free chiral multiplet with mass deformation \eqref{eqn:mass1} is perturbatively trivial.
Suppose $m_D \ne 0$.
The the holomorphic twist of two free chiral multiplets with mass deformation \eqref{eqn:mass2} is also perturbatively trivial. 
\end{prop}

Let's move on to the holomorphic twist of $\cN=2$ theories.
A hypermultiplet is defined for any symplectic representation of the gauge group~$U$.
The holomorphic twist of the hypermultiplet valued in $U$ is the $\beta\gamma$ system valued in $U$.
This is the twisted version of the claim that in $\cN=1$ terms, a hypermultiplet consists of a chiral multiplet plus an anti-chiral multiplet.
Hypermultiplets can have mass.
The mass terms which arise from the holomorphic twist are always of `Dirac' type \eqref{eqn:mass2}.
Generally, a hypermultiplet mass exists for any $M \in \op{Sp}(U)$ and takes the form \eqref{eqn:lagmass}.
(More generally, a superpotential $W \in \cO(U)$ for hypermultiplets can be turned on if it is invariant with respect to the symplectic group).

A word of \ul{\textbf{warning}}.
When physicists speak of ``hypermultiplets valued in a representation $R$'', they almost always mean hypermultiplets valued in the symplectic representation $T^* R = R \oplus R^*$ as I defined in the previous paragraph.

Let's now couple to vector multiplets.
The holomorphic twist of the $\cN=1$ vector multiplet valued in $\lie{g}$ is BF theory to the local Lie algebra $\Omega^{0,\bu}(\C^2, \lie{g})$.
The holomorphic twist of the $\cN=2$ vector multiplet valued in $\lie{g}$ is BF theory to the local Lie algebra $\Omega^{0,\bu}(\C^2, \lie{g})$ plus the $\beta\gamma$ system valued in the adjoint representation $\lie{g}$.
This is the twisted version of the fact that the $\cN=2$ vector multiplet consists of a $\cN=1$ vector multiplet plus a $\cN=1$ chiral multiplet.

We can couple the $\cN=2$ vector multiplet to any symplectic $\lie{g}$-representation.
From this point of view, there is always a superpotential around.
Let $(A,B)$ be the fields of BF theory valued in $\lie{g}$ and let $(\phi ,\psi)$ be the fields of the adjoint-valued $\beta\gamma$ system; together these fields comprise the twist of the $\cN=2$ vector multiplet.
Let $(\beta,\gamma)$ denote the fields of the $\beta\gamma$ system valued in the symplectic representation $U$.
Then there is a superpotential of the form
\beqn
W = \gamma \phi \gamma .
\eeqn

Notice that in the $\cN=2$ vector multiplet there is a $\cN=1$ chiral multiplet valued in the adjoint representation whose twisted fields we called 
\beqn
\phi \in \Omega^{0,\bu}(\C^2, \lie{g}) , \quad \psi \in \Omega^{2,\bu}(\C^2, \lie{g}^*)[1] .
\eeqn
Turning on a mass for this chiral multiplet is not allowed by $\cN=2$ supersymmetry.
The sort of mass term one considers, at the level of the twist, is
\beqn
W_\mu = \mu \op{Tr}(\phi^2) 
\eeqn
where $\mu \in \C$.

At the twisted level, there is an easy way to see why this mass terms breaks symmetry.
The twist of $\cN=2$ supersymmetry has a residual global symmetry which assigns the gauge field $A$ weight zero and $\phi$ weight one.
This term clearly breaks this symmetry.

I'm not sure what you had in mind by starting with a $\cN=2$ theory and softly breaking to $\cN=1$ with such a mass term. Can you elaborate?

\subsection{Masses in Seiberg duality}

You also mentioned masses in Seiberg duality.
To me, the obvious thing to turn on a mass term in the electric theory and follow it through the duality.
What sort of deformation of the magnetic theory arises?

Consider holomorphic QCD with gauge group $SU(N)$ and $F$ flavors.
The (anti)quarks become $\beta\gamma$ systems whose fields we denote
\beqn
Q \in \Omega^{0,\bu}(\C^2) \otimes V^{\oplus F} , \quad \til Q \in \Omega^{0,\bu}(\C^2) \otimes (V^*)^{\oplus F} .
\eeqn
Here $V = \C^N$ is the fundamental $SU(N)$ representation.

Pick a basis for $\C^F$ and write corresponding vectors in $V^{\oplus F}$ as $(v_1,\ldots,v_F)$, where $v_i \in V$ for $i=1,\ldots,F$.
We consider the mass deformation
\beqn
W_{electric,m} = m Q_1 \til Q_1 .
\eeqn
Note we consider a Dirac type mass (this is the only one allowed).
An elaboration of the proof of proposition \ref{prop:mass} shows the following.

\begin{prop}
Suppose $m \ne 0$.
Holomorphic QCD with gauge group $SU(N)$ with $F$ flavors with the mass deformation $W_{electric} = m Q_1 \til Q_1$ turned on is equivalent to holomorphic QCD with gauge group $SU(N)$ with $F-1$ flavors.
\end{prop}

The magnetic theory is holomorphic QCD with gauge group $SU(F-N)$ with $F$ flavors and a meson valued in $\C^{F^2}$.
The quarks will be denoted $q, \til q$ and the meson $M = (M_{ij})$.
Before any masses, the superpotential is cubic
\beqn
W_{magnetic,0} = q M \til q .
\eeqn
Turning on the mass in the electric theory corresponds to deforming this superpotential by
\beqn
W_{magnetic,0} \rightsquigarrow W_{magnetic,m} = q M \til q + m M_{11} .
\eeqn
Note that there is no `mass' term in the magnetically dual theory.
Turning on the mass in the electric theory \eqref{eqn:emass} corresponds to turning on the linear superpotential $m M_{11}$ in the magnetic theory.

This linear term breaks the magnetic gauge group to $SU(F-N) \rightsquigarrow SU(F-N-1)$ and removes a flavor.

\begin{prop}
Suppose $m \ne 0$.
Holomorphic QCD with gauge group $SU(F-N)$ with $F$ flavors with the superpotential $W_{magnetic,m} = q M \til q + m M_{11}$is equivalent to holomorphic QCD with gauge group $SU(F-N-1)$ with $F-1$ flavors.
\end{prop}

Note that this is exactly the expected dual theory!


\subsection{Konishi anomaly}

The Konishi anomaly is very simple in the holomorphic twist.
Consider holomorphic QCD with gauge group $G$ and matter valued in $V$.
The action is
\beqn
S_{hQCD} = \int B F_A + \int \beta \dbar_A \gamma .
\eeqn
There is a classical $\C^\times$ symmetry which weights $\gamma$ with weight $+1$ and $\beta$ with weight $-1$.
There is a holomorphic refinement of this symmetry by the abelian local Lie
$\Omega^{0,\bu}(\C^2)$.
For $\alpha \in \Omega^{0,\bu}(\C^2)$ the associated local Noether current is
\beqn
J_\alpha = \int \beta \alpha \gamma .
\eeqn

The Konishi anomaly is a mixed $\cK$-gauge anomaly which represents the failure of the local symmetry by $\cK$ to persist at the quantum level.
It's computed by the triangle diagram with a single $\alpha$ leg and two gauge field legs.
It is represented by the local functional
\beqn
\int \alpha \op{tr}(\del A \del A) .
\eeqn

Now, suppose we have turned a superpotential on.
No longer does $J_{\alpha}$ determine a classical symmetry.
The ``classical anomaly'' to this is represented by the local functional
\beqn
\int \alpha \gamma \del W = \int \alpha \gamma^i \frac{\del W}{\del x^i} (\gamma) .
\eeqn

There is a generalization of this anomaly.
Consider the Lie algebra $\op{Vect}(V)$ of vector fields on $V$ (maybe formal or algebraic).
Then we can form the local Lie algebra
\beqn
\cK \define \Omega^{0,\bu}(\C^2) \otimes \op{Vect}(V) .
\eeqn
We will write sections as 
\beqn
\alpha \otimes X = \alpha \otimes f^i \frac{\del}{\del x^i}
\eeqn
(In the above terminology we were looking at the stupid abelian subalgebra of elements $\alpha \otimes Eu$ where $Eu = x^i \frac{\del}{\del x^i}$ is the Euler vector field on $V$.)
Holomorphic QCD admits the following very natural $\cK$-background 
\beqn
J_{\alpha \otimes X} = \int \alpha \<\beta, X(\gamma)\> = \int \alpha \beta_i f^i(\gamma) .
\eeqn 

\begin{prop}
The one-loop anomaly for $J$ to prescribe a quantum $\cK$-background for holomorphic QCD is represented by the local functional
\beqn
\int \alpha (\op{div} X)(\gamma) \op{Tr}_V(\del A \del A) = \int \alpha \frac{\del f^i}{\del x^i} (\gamma) \op{Tr}_V(\del A \del A). 
\eeqn
\end{prop}

Geometrically, there is the following interpretation of this anomaly.
Recall that holomorphic QCD is the cotangent theory to the derived moduli space of holomorphic maps
\beqn
\C^2 \to [V \slash G] .
\eeqn
The generalized Konishi anomaly is the anomaly for the quantization of the universal symmetry of vector fields on the target $[V\slash G]$.

\subsection{Compactification on an elliptic curve}

You mentioned understanding how compactficiations look over the moduli of some geometry.
Let me take the moduli of elliptic curves.
The compactification of any of the theories associated to the holomorphic twist of a four-dimensional $\cN=1$ theory on an elliptic curve is the holomorphic twist of a two-dimensional theory with $\cN=(2,2)$ supersymmetry.
Naively, the compactified factorization algebra only depends on the cohomology of the elliptic curve. 
That doesn't look so interesting over the moduli space, so I'm not sure what you had in mind.
Can you elaborate?
Could you give me a hint of the sort of feature(s) you are trying to see?

\section{chiral rings}

Let $\lie{g}$ be a simple Lie algebra and fix an invariant form on $\lie{g}$ so that we fix an isomorphism $\lie{g} = \lie{g}^*$.
Consider the affine supermanifold
\beqn
V = \Pi \lie{g} \oplus \Pi \lie{g} ,
\eeqn
whose ring of functions is $R = \cO(V) = \wedge^\bu \left(\lie{g} \oplus \lie{g}\right)$.
The chiral ring is functions on the sub supermanifold cut out by the equations
\beqn
X^2 = 0, XY + YX = 0, Y^2 = 0 \in S^2(\lie{g}) .
\eeqn
In other words, we have $R = \C[X,Y]_{X,Y \in \lie{g}}$ where we consider each $X,Y \in \lie{g}$ as an odd generator.
We then consider the ideal $I$ generated by 

We also consider $\lie{g}$ as a (super) manifold.
To write a map of affine supermanifolds $\lie{g} \to V$
is to write a map of commutative superalgebras
\beqn
R = \wedge^\bu (\lie{g} \oplus \lie{g}) \to S^\bu (\lie{g}) .
\eeqn
Since $R$ is a free superalgebra it suffices to prescribe the map on the (odd) linear generators.
Define
\beqn
\{-,-\} \colon \lie{g} \oplus \lie{g} \to S^2 \lie{g} \subset S^\bu (\lie{g}) 
\eeqn
by $\{X,Y\} = XY + YX$.
Thus, $\{-,-\}$ determines a map of affine supermanifolds
\beqn
\{-,-\} \colon \lie{g} \to V .
\eeqn

\section{$SU(2)$ QCD with three flavors}

We study $SU(2)$ QCD with $F$ flavors.
This means that the quarks live in $(\C^2)^F \oplus (\C^2)^{* F} = (\C^2)^F$ where $\C^2$ is the fundamental $SU(2)$ representation.
We have used the fact that the fundamental and anti-fundamental representations of $SU(2)$ are isomorphic.
We will write quark fields as $Q^1,\ldots,Q^{2F}$.
The global symmetry group is
\beqn
SU(2F) \times U(1)_R .
\eeqn
For the $R$-symmetry to be non-anomolous one can take each quark to have $R$-charge~$R_{micro} = \frac{F-2}{F}$.

The classical moduli space is the following quadric
\beqn
\cM_{cl} = \{v \; | \; v \wedge v = 0\} \subset \wedge^2 \C^{2F} .
\eeqn
Identifying $v \in \wedge^2 \C^6$ with a $6 \times 6$ skew-symmetric matrix, we have the following simplifications:
\begin{itemize}
\item When $F = 2$ the defining equation is simply $\op{Pf}(v) = 0$.
\item When $F = 3$ the defining equation is $(v^{-1})_{ij} \op{Pf}(v) = 0$.
\end{itemize}

What if we add masses?
Let $\bm = (\bm_{ij}) \in \lie{so}(2F)$ be a skew-symmetric matrix.
We consider the mass term
\beqn
W = \frac12 \bm Q Q = \sum_{i < j} \bm_{ij} Q^i Q^j .
\eeqn

\subsection{$F = 3$}
As a function of the mass matrix, the quantum moduli space $\cM(\bm)$ is expected to be deformed as
\beqn
v^{-1} \op{Pf}(v) = \bm ,
\eeqn
or in components $(v^{-1})_{ij} \op{Pf}(v) = \bm_{ij}$.
From this, we see that in the limit $\bm \to 0$ that the quantum moduli space agrees with the classical moduli space, at least away from the singular points.

\section{October 22, 2024}

Consider the holomorphic twist of a free 4d $\cN=1$ chiral multiplet with $R$-charge $r$.
This theory exists on $\C \times \PP^1$; I claim its twist is equivalent to the higher dimensional $\beta\gamma$ system on $\C \times \PP^1$ where $\gamma$ is a $(0,\bu)$-form with values in the restriction of the holomorphic line bundle $\cO(-r)$ on $\PP^1$.
That is the fields are
\begin{align*}
\gamma & \in \Omega^{0,\bu}(\C \times \PP^1 , \cO(-r)) \\
\beta & \in \Omega^{2,\bu}(\C \times \PP^1, \cO(r))[1] ,
\end{align*}
and the action is $\int_{\C \times \PP^1} \beta \dbar \gamma$.

\begin{prop}
The compactification of the holomorphic twist of the free chiral of $R$-charge~$r$ along $\PP^1$ is equivalent to the free $\beta\gamma-bc$ system
with coefficients in $H^\bu(\PP^1, \cO(-r))$.
\end{prop}

In particular:
\begin{itemize}
\item if $r <1$ then we have $1-r$ ordinary $\beta\gamma$ systems on $\C$.
\item if $r =1$ then this is the trivial theory.
\item if $r > 1$ then we have $r-1$ ordinary $bc$ systems on $\C$.
\end{itemize}

More generally, one can consider the chiral multiplet on a $\PP^1$-fibration over a Riemann surface $\Sigma$.
In this way, I think we can make sense of the partition function of the four-dimensional theory placed on `ruled surfaces of genus $g$'---I think these are nothing but $\PP^1$-fibrations over a genus $g$ Riemann surface. 
For genus 1 such manifolds have a two-parameter moduli (roughly: the complex structure on the base elliptic curve and the parameter describing how the $\CP^1$'s are glued).

One can also generalize this compactification to include a background $U(1)$ gauge field with nontrivial flux through $\PP^1$.
In this case, the resulting compactification is the $\beta\gamma-bc$ system with coefficients in $H^\bu(\PP^1, \cO(-r-m))$ where $m = \int_{\PP^1} F$ is the flux through $\PP^1$.

\end{document}
